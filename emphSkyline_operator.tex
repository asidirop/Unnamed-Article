\emph{Skyline operator }

As discussed in the introductory section, given a set of attributes that
characterize scientific performance, the \emph{skyline operator} outputs
the ones that cannot be surpassed by any other scientist in the dataset.
The concept of skyline, calculated by the respective operator , has been
utilized in the field of Computer Science for decades and dates back to
the definition of the \emph{Pareto frontier} in economics . However, the
skyline set does not refer to efficient resource allocation; rather it
provides a multi-criteria selection of distinguished scientists. The
algorithm by Chomicki et al. known as Sort-First-Skyline (SFS) was
employed in to experimentally verify the identified elite groups of
scientists and it will be utilized in the present analysis as well, due
to its minimal computational cost and efficiency.

To appropriately identify the dimensions that will serve as attributes
to the skyline calculation various experiments were performed in and .
The results of this analysis comply with literature in the sense that
there exist groups (clusters) of highly correlated scientometric
indicators and therefore the skyline does not vary significantly with
different combinations of dimensions, as long as they are derived from
the same group of indices. In Table 1 we present the Spearman's
correlation coefficients between selected rank methods, based on various
scientometric indicators. For the experiments conducted in the present
work, we selected the \emph{h}-index as a ranking method because it is
the most commonly used performance indicator. We also selected the
Perfectio­nism Index because it is the most dissimilar index with
\emph{h} -index based on Table 1. Finally, the \emph{A}-index was
included, since it is also dissimilar with \emph{h}-index and offers a
different counting of citations in the \emph{h}-core.

% First macro on next line not (yet) supported by LaTeXML: 
% \protect\hypertarget{_Ref477860026}{}{}Table 1. Rank Methods correlation
based on Spearman's coefficient
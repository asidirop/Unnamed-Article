One can argue that ranking into levels produces a lot of ties. This
claim is not true. As it is shown in Figure 6, \emph{h}-index, which is
the most commonly used method, produces much fewer levels and much
greater number of ties. The peak of \emph{h}-index cardinality is more
than 12000 while in RR the worst case (Figure 5) is less than 1000.

Finally, Table 3 illustrates that all variations of \emph{RR} are not
necessarily similar with their gene­rators. For example, although
\emph{PI} is one of the generating dimensions of \emph{RR(d1),} the
coefficient between them is only 0.307. This result can be attributed to
the interconnected nature of bibli­ometric indicators, meaning that the
resulting relative ranking of a scientist may significantly differ from
her/his individual ranking on each one of the generator dimensions.
However, the information conveyed by the RR-index provides the relative
ranking as a result of overall performance, including different aspects
of scientific output as expressed by the generator dimensions.
Consequently, the RR ranking can be considered a more unified and
repre­sentative evaluation metric, compared to its individual generator
dimensions.

Table 5 illustrates the Rank Table according to \emph{RR(d1).} We have
included columns \emph{h}, \emph{A} and \emph{PI} which are the
generators of \emph{RR(d1)} as well as the values for \emph{C} (number
of citations), \emph{P} (num­ber of publications) and \emph{C/P}
(citations per publication). The last column shows the skyline level to
which each respective scientist has been assigned.

The first 13 scientists were ranked in the first skyline level and they
are assigned the same value for \emph{RR(d1)}. In this list of top
ranked researchers, we encounter scientists who can be grouped into two
subsets based on their work; one group is comprised of those who have
worked in core computer science (e.g., networking, compilers,
databases), and the second group of those who have contributed to the
field of computational methods for biology, with the latter group being
the largest of the two. More specifically, in the former group, we see
Scott Shenker well known for his contributions in networking theory and
practice, Ian Foster of the community of high performance computing
(grid and cloud computing), and Jeffrey Ullman whose work spans across
several research areas (compilers, programming languages, databases). In
the latter group, David Haussler was a member of the team who sequenced
the human genome, Robert Tibshirani made solid contributions to
statistical learning theory and its application to biological problems,
Altschul Stephen, David Lipman and Web Miller, co-developers of the
well-known BLAST family of tools for sequence comparison. Additionally,
there is Higgins Desmond, developer of the Crystal-W and Crystal-X tools
for multiple se­quence alignment, Gish Warren co-worker of David
Haussler to human genome sequencing, and of Stephen Altschul to the
development of BLAST.

In the second skyline level, we mainly encounter core computer
scientists, namely Hector Garcia-Molina of databases, Deborah Estrin of
embedded systems (sensors), David Culler of networking, Simon Herbert of
political science and economics, Ronald Rivest of crypto­graphy and
co-developer of the RSA cryptosystem, Vladimir Vapnik the father of
statistical learning theory, Thomas Cormen well-known for his work on
distributed algorithms, Claude Shannon the father of information theory,
and so on. Finally, Eugene Myers a computational biologist who
co-authored the famous Science-Nature paper on the sequence of the human
genome is also ranked at the same skyline level. The reason why there is
a large number of computational biologists/bioinformatics researchers in
the top ranked group could be the recent popularity and fast growth of
the field, as compared to the more ``old-fashioned'' domains of core
computer science. It can be observed that even though this group may
have scored relatively lower values in \emph{h}-index, the values of A
and C/P are significantly higher, meaning that these bioinformatics
researchers accumulate citations at a faster pace compared to the more
mature researchers of the core computer science group. As a result,
different publishing patterns can be identified and rewarded using the
RR-index.
In Figure 4 we doubled the dimensions used in \emph{RR(d1).} For each
one of the three dimensions of \emph{RR(h, A, PI)} an additional one was
introduced that displays a high correlation with one of the original
dimensions based on Table 1. The rationalized
\emph{h\textsubscript{ra}} \textsubscript{t}-index was selected due to
its high similarity to \emph{h}-index, \emph{e}-index as it is similar
to \emph{A}, whereas \emph{h\textsubscript{norm}} was included because
it is correlated with \emph{PI}. As a result, the skyline levels
decreased and the skyline sizes increased. We see exactly the same
behavior in Figure 5. As expected, the more dimensions we include the
more members are placed in each skyline level. We observe a denser
ranking in moderate skyline levels, meaning that with the addition of
correlated dimensions the segmentation in performance levels is less
detailed due to the similarity in rankings produced by correlated
indices. As depicted at level 40, there is a peak at 900 skyline
members. In Table 3 and Table 4 the correlation values indicate that all
four variations of RR-index produce similar overall ranking, even though
the respective sizes of individual skyline levels may differ. This means
that if representative rank methods are selected, then there is no need
for a large number of them to be used in producing a unified
representative ranking. Additionally, when less correlated dimensions
are selected the resulting segmentation becomes more detailed and
precise.
\emph{Rainbow-Ranking (RR-index)}

The skyline operator selects the best performing scientific entities
based on multiple criteria, but does not assign a meaningful and
comparable ranking score to every scientist. The skyline operator, given
a set of scientists, just extracts an elite set. Therefore
\emph{Rainbow-Ranking} is in­troduced to apply the skyline operator
iteratively until all scientists of a dataset are classified into a
skyline level. More specifically, given a set of scientists
\emph{A=X\textsubscript{1}} , the first call of skyline produces the
first skyline level. We denote this first set of scientists as set
\emph{S\textsubscript{1}} . In the next step, we compute set
\emph{X\textsubscript{2}=X\textsubscript{1}-S\textsubscript{1}} , which
contains the scientists in the dataset that were not classified in the
first skyline set \emph{S\textsubscript{1}}. For the set
\emph{X\textsubscript{2}} the skyline operator is applied once more and
the result is the second skyline level (\emph{S\textsubscript{2}}). The
process continues until all the scientists of the dataset are assigned a
value that corresponds to the skyline level they have been ranked in. It
is obvious that the set \(S_{i}\) is dominating over \(S_{j}\)
(\(S_{i} \prec S_{j}\)) if \emph{i\textless{}j.} Also, for researchers
\emph{α} and \emph{β} it holds that α\(\prec \beta\) if
% First macro on next line not (yet) supported by LaTeXML: 
% \(\alpha \in S_{i}\) and \(\beta \in S_{j}\) and\(\ S_{i} \prec S_{j}\).
Figure 1 shows a graphical representation of the skyline levels with two
dimensions (featu­res): citations per publication and the
\emph{h} -index. We have selected the aforementioned subset of our
original dataset containing 700 researchers from Greek Universities and
have ranked them according to these two dimensions. Every point in
Figure 1 corresponds to a scientist. Each li­ne connecting the points
corresponds to a different skyline level. The x-axis represents rank­ing
positions of each scientist according to their \emph{h}-index, whereas
on the y-axis the respective ranking positions according to citations
per publication. Since this iterative procedure results into a plot with
grouped curves as shown in Figure 1, and the procedure is built over the
no­tion of skyline, we have selected the name \emph{Rainbow Ranking}.
Also, in the above plot (and in all our experiments), in case of a tie,
the average position of all tie members is used to express the rank
position. This is the fairest method, as well as the method that is used
in Spearman's coefficient computation. Note that \emph{h-}index produces
a lot of ties.

Inevitably, a score value should be produced for each rank level. If
this score simply represent the level number, it would provide limited
interpretability for the relative ranking of each scientist compared to
his peers; therefore, a normalization of this value is required. To
sum­marize the ranking levels into a single number metric, given a set
of scientists \emph{A} and a set of dimensions \emph{dims}, we define
the RR-index of a scientist based on \emph{dims} as follows:

% First macro on next line not (yet) supported by LaTeXML: 
% \(\text{RR}\left( \text{dims} \right) = 100*\left( \frac{\left| A_{\text{above}}(a,\ dims) \right|}{\left| A \right|} + \frac{\left| A_{\text{tie}}(a,dims) \right|}{2*\left| A \right|} \right)\)
(1)

In Equation (1) \textbar{}\emph{A}\textbar{} is the total number of
scientists in our dataset,
\textbar{}\emph{A\textsubscript{above}(a,dims)}\textbar{} is the number
of scientists ranked at higher skyline levels than scientist \emph{a}
based on dimensions \emph{dims}. Note that level 1 is considered higher
than level 2 in a rank table. Additionally,
\textbar{}\emph{A\textsubscript{tie}(a,dims)}\textbar{} is the number of
scientists who are ranked at the same level with scientist \emph{a},
excluding scientist \emph{a.} Consequently, the following holds for the
RR-index:

% First macro on next line not (yet) supported by LaTeXML: 
% \(0 < RR(dims) \leq 100\) (2)

The case when \emph{RR(dims)=100} means that scientist \emph{α} is
ranked in the first skyline level alone. Since all the members of a
skyline level should be assigned with the same score, we have chosen to
assign a score analogous to the average rank position of all tie members
normalized to the range 0-100.

The key components for the calculation of the RR-index are the skyline
dimensions and the ranking positions assigned according to each
dimension. By selecting different bibliometric indices as skyline
dimensions, the calculated RR-index can be fully customizable. However,
since bibliometric indices are highly correlated with each other, as
depicted in Table 1, se­lecting highly correlated indices would yield
analogous results in the final skyline ranking. As the number of
dimensions (criteria) increases the skyline's sizes increases as well
because the number of individuals that are not clearly bypassed by
others increases.

In the next section we present our experiments with multiple dimension
combinations and investigate the distinguishing power of the RR-index.
A Rainbow at the Skyline after\\
the Storm of Indicators for Ranking Scientists

Georgios Stoupas\textsuperscript{1}, Antonis
Sidiropoulos\textsuperscript{2}, Antonia Gogoglou\textsuperscript{3},\\
Dimitrios Katsaros\textsuperscript{4}, Yannis
Manolopoulos\textsuperscript{3}

\textsuperscript{1} grgstoupas@gmail.com, \textsuperscript{2}
asidirop@it.teithe.gr

Department of Information Technology, Alexander Technological
Educational Institute of Thessaloniki, Thessaloniki, Greece

\textsuperscript{3} \{agogoglou, manolopo\}@csd.auth.gr

Department of Informatics, Aristotle University, Thessaloniki, Greece

\textsuperscript{4} dkatsar@inf.uth.gr

Department of Electrical \& Computer Engineering, University of
Thessaly, Volos, Greece

Abstract

Various scientometric indices have been proposed in an attempt to
express the quantitative and qualitative characteristics of scientific
output. However, fully capturing the performance and impact of a
scientific entity (author, journal, institution, etc.) still remains an
open research issue, as each proposed index focuses only on particular
aspects of scientific performance. Therefore, scientific evaluation can
be viewed as a multi-dimen­sional ranking problem, where dimensions
represent the assorted scientometric indices. To address this problem,
the skyline operator has been proposed in with multiple combinations of
dimensions. In the present work, we introduce a new index derived from
the utilization of the skyline operator, called Rainbow Ranking or
RR-index that assigns a category score to each scientist instead of
producing a strict ordering of authors. Our RR-index allows for the
combination of any known indices depending on the purposes of the
evaluation and outputs a single number metric expressing multi-criteria
relative ranking. The proposed methodology was experimentally evaluated
using a dataset of over 105,000 scientists from the Computer Science
field.

Conference Topic\\
Indicators